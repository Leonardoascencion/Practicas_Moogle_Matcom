\documentclass[100t]{article}
\usepackage[spanish]{babel}

% Title Page
{{\Huge  \title{\textbf{Informe}}}
{\large \author{Leonardo Peláez Ascención}}

\begin{document}
\maketitle
\section{Introducción}
En este documento hablaré brevemente sobre los codigos esenciales implementados en mi proyecto de Moogle! los cuales estan destinados a en conjunto realizar la tarea explicada en la presentación general que se encuentra junto con ese documento.
\subsection{Aclaración importante}
Este proyecto no se encuentra terminado, está tan solo en la versión 0.1.5 de su desarrollo. No obstante seguiré adelante comentando lo que hay en el proyecto implementado y aparte lo q hablaré lo que tengo pensado implementar a futuro. Obviamente, cada vez que se actualice el código y se acerque a su estado final,este documento será editado agregando las funcioalidades existentes.
\section{Implementado}
Esta sección hablará solo de los codigos implementados.
\subsection{Query}
Por muy sencillo que sea de implementar se encuentra, recibe una entranda del usuario y la almacena para luego usarla en la busqueda del documento mas relevante.
\subsection{Matriz asociada a los archivos}
En las primeras lienas del código esta implementado una manera de almacenar todos los documentos en una especie de matriz. Conformado por diccionarios asocia a cada título que encuentre su contenido respectivo.
\section{No implementado}
\subsection{Matriz TF-IDF}
Se deberia crear una matriz donde a cada archivo se le asigne un valor de TF en cada documento, y un valor de IDF(su significado y como se calcula se encuentran en la presentación).
\subsection{Snippet}
Este código una vez devuelto el documento mas relevante, el programa muestra una porción del mismo.
\subsection{Sugerencias}
Aquí el programa se ejecuta en caso de que la query no pueda devolver ningún documento, en ese caso este se encargará de suegerir la búsqueda  de una frase o palabra parecida a la query la cual si aparecerá en uno o algunos de los documentos. 
\section{Nota Final}
Creo que estas son las partes mas importantes del codigo en cuestión. Si me falta alguna seguramente lo añadiré en una actualización. 
\linebreak
{\LARGE \textbf{\textit{Nuevamente resaltar mis disculpas por un proyecto incompleto y fuera de tiempo. Y en caso de que sí hay alguien leyendo, gracias por leer hasta el final.}}}




\end{document}
